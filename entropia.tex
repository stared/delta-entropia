\documentclass[10pt,a4paper]{article}
\usepackage[utf8]{inputenc}
\usepackage[T4]{fontenc}
\usepackage[polish]{babel}
%\usepackage {polski}
\let\lll\undefined
\usepackage{amssymb,amsmath,amsthm}
\marginparwidth=5.2cm \hoffset=2.5cm \textwidth=12.5cm
\textheight=25.2cm \reversemarginpar \voffset=-1in
%\addtolength{\textheight}{2in}

% Moje pliki latexowe artykułów, które napisałem mają mniej więcej 1100 - 1600 słów
% i to jest trochę ponad 2 strony. A więc celuj w 1000 słów (lub 500 - na 1 stronę), a jak przekroczysz
% to nic się nie stanie. Objętość tekstu nie jest taka ważna, ważniejsza jest treść ;).

\begin{document}
%tytuł
\noindent\textbf{\LARGE Prawdopodobieństwo a informacja}

\medskip
%autor
\noindent\textit{\Large Piotr Migdał} \marginpar{\footnotesize
*dr z ICFO--The Institute of Photonic Sciences, Castelldefels (Barcelona)}

\medskip

O ile entropia jest używana w języku potocznym jako chaos i nieuporządkowanie, to sama wielkość jest ściśle określonym pojęciem --- \emph{informacją Shannona}:
%
\begin{align}
    H = -\sum_{i=1}^{n} p_i \log(p_i),
\end{align}
%
gdzie $\{p_1, \ldots, p_n\}$ to pewien rozkład prawdopodobieństwa. Będziemy używać podstawy logarytmu $2$, co odpowiada mierzeniu entropii w \emph{bitach}. 

    \marginpar{\footnotesize Korzystatnie z innej podstawy logarytmu, np. $e=2.718\ldots$ przemnoży wynik przez stałą, a zatem odpowiada tylko zmianie jednostek: $\ln(x)=\ln(2)*\log_2(x)$.}

Gdy nasz rozkład składa się z jednej możliwości, entropia wynosi $-0 \log(1) = 0$ --- wszak nie ma tu miejsca na losowość. Gdy rzucamy uczciwą monetą, entropia to $-\tfrac{1}{2} \log(\tfrac{1}{2}) -\tfrac{1}{2} \log(\tfrac{1}{2}) = 1$.

    \marginpar{\footnotesize Z matematycznego punktu widzenia $\lim_{p \to 0} p \log(p) = 0$, z fizycznego --- nie chcemy by dodanie nierealistycznej opcji (tj. z zerowym prawdopodobieństwem) zmieniało wynik.}

Entropia jest największa, gdy dla ustalonej liczby zdarzeń wszystkie są równo prawdopodobne --- wtedy entropia to $\log(n)$.

I co robi w tym wzorze minus oraz logarytm?


\section{TO DO:}

\begin{itemize}
    \item Prawdopodobieństwo a informacja
    \item Kodowanie
    \item Gramy w 20 pytań!
    \item Temperatura?
\end{itemize}

\section{Kolejne odcinki}

\begin{itemize}
    \item Informacja wzajemna
    \item Inne entropie
\end{itemize}


 \marginpar{\footnotesize Dygresja}


 Lektury:


\end{document}
