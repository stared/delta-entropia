\documentclass[10pt,a4paper]{article}
\usepackage[cp1250]{inputenc} %cp1250 standard kodowania polskich znakow pod windows
%\usepackage[latin2]{inputenc} %latin2(inaczej ISO-8859-2) standard kodowania polskich znakow czesciej spotykany pod linuxem

\usepackage {polski}
\usepackage{amssymb,amsmath,amsthm}
\marginparwidth=5.2cm \hoffset=2.5cm \textwidth=12.5cm
\textheight=25.2cm \reversemarginpar \voffset=-1in
%\addtolength{\textheight}{2in}

\begin{document}
%tytu�
\noindent\textbf{\LARGE Tytu� artyku�u}

\medskip
%autor
\noindent\textit{\Large Imi� NAZWISKO*} \marginpar{\footnotesize
*afiliacja}

\medskip
 Polskie litery ����󜿟 ��ʣ�ӌ�� Polskie litery mo�na
te� wprowadza� w konwencji "ciachowej" np./a ale najpierw trzeba
u�y� komendy $\backslash$prefixing \prefixing.
 \marginpar{\footnotesize Niekt�re wyja�nienia
dla bardziej zaawansowanych czytelnik�w lub jakie� inne dygresje
mo�na umieszcza� na marginesie. Wi�kszo�� rysunk�w te� tu l�duje.}
Najlepiej j� wpisa� przed $\backslash$begin \{document\}. Teraz mamy
/a itd. Je�li chcemy umie�ci� w tek�cie "ciach" to musimy u�y� dw�ch
ciach�w
//. Konwencj� "ciachow�" mo�na wy��czy� komend� $\backslash$nonprefixing \nonprefixing.

\end{document}
